% !TeX spellcheck = da_DK
\chapter{Introduktion}
Apopleksi er den tredje største dødsårsag i Danmark og ca. 12.500 personer indlægges hvert år pga. sygdommen \cite{Hjernesagen2015a}. I 2011 levede 75.000 danskere med følger af apopleksi, og ud af disse er omkring hver fjerde person afhængig af andres hjælp \cite{Hjernesagen2015a}. Det er dog ikke alle patienter, der får mén af apopleksi. \\
Der sker en stigning af indlæggelsesforløb for mænd og kvinder, når de bliver ældre end 65 år \cite{Sundhedsstyrelsen2011}. Antallet af danskere, der lever med følger og varige mén af apopleksi, forventes derfor at være stigende i takt med, at der kommer flere ældre \cite{Sagen2014}. Apopleksi er i forvejen den sygdom, der kræver flest plejedøgn i sundhedssektoren, men i takt med, at der kommer flere ældre, forventes det at udgifterne til denne pleje stige. Ud fra et økonomisk perspektiv er det derfor omkostningsfuldt for samfundet ift. behandling, rehabilitering og produktivitetstab.  Udgifterne til sygdommen udgør 4\% af sundhedsvæsenets samlede udgifter \cite{Hjernesagen2015a, Kruuse2014}.

Følgerne af apopleksi opstår ofte pludseligt og kan medføre både fysiske og mentale konsekvenser for patienten \cite{Muus2008}. Efter et apopleksitilfælde kan patienter opleve nedsat eller ikke funktionsdygtig balance. Problemer med balancen opstår, da encephalon ikke kan bearbejde de balanceinformationer, som proprioceptorerne og sansereceptorerne sender. \cite{Karnath2003} Dette resulterer i, at 40\% af det samlede antal apopleksipatienter oplever faldulykker i det første år. \cite{Association2006}. 
Balanceproblemer har alvorlige konsekvenser for apopleksipatienter, da de bl.a. kan føre til begrænsninger i hverdagen. \cite{Muus2008,Nichols1997} For en apopleksipatient med balanceproblemer kan det være vanskeligt at vende tilbage til sin normale hverdag, da almindelige huslige pligter, såsom rengøring og personlig pleje kan være vanskeligt at klare uden hjælp. \cite{Sundhedsstyrelsen2010}

Balanceproblemer samt begrænsninger i hverdagen kan medføre nedsat livskvalitet. Dette ses eksempelvis ved, at apopleksipatienter har dobbelt så stor selvmordsrate som baggrundsbefolkningen \cite{Sundhedsstyrelsen2010}. Et apopleksitilfælde medfører en pludselig afbrydelse i patientens livsforløb. Det kan for patienten blive uoverskueligt at opretholde sociale- og familierelationer, hvilket medfører, at de senere i livet oplever en forringelse af deres livskvalitet. En forbedret livskvalitet kan skabes ved hurtigere rehabilitering samt forbedrede kropslige funktioner, herunder balancen. \cite{Sundhedsstyrelsen2010}

For at apopleksipatienter opnår den bedst mulige behandling og rehabilitering er det afgørende, at der er et fungerende sammenspil mellem kommuner, sygehuse og praktiserende læger \cite{Sundhedsstyrelsen2010}. Det er essentielt, at rehabiliteringen påbegyndes få dages efter apopleksitilfældet er opstået, for så vidt muligt at genskabe tabte funktionsevner. \cite{Kruuse2015}

\section{Initierende problem}
Hvilke fysiologiske konsekvenser kan apopleksi have for patienten, og hvad er rehabiliterings mulighederne for en patient med balanceproblemer?  