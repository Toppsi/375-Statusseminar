% !TeX spellcheck = da_DK
\chapter{Problemløsning}
I dette afsnit vil projektets problemformulering fremgå. Derudover beskrives de funktionelle krav for systemet, dets opbygning samt kravspecifikationer.

\section{Problemformulering}
Hvordan designes et biofeedbacksystem med et accelerometer således, at det hjælper apopleksipatienter under rehabilitering af balancen?
%Apopleksi er en sygdom, som har indvirkning på blodgennemstrømningen til encephalon, da den nedsætter blodtilførslen enten ved en blodprop eller ved en blødning \cite{Hjernesagen2015a}. Et apopleksitilfælde kan være forårsaget af enten en embolia cerebri (iskæmisk) eller hæmorrhagia cerebri (hæmoragisk). \cite{Ritter2015} Sundhedsstyrelsen definerer apopleksi som pludseligt opståede fokalneurologiske symptomer af formodet vaskulær genese med en varighed på over 24 timer.\cite{Sundhedsstyrelsen2009} Hvis varigheden er under 24 timer, betegnes det som transitorisk cerebral iskæmi (TCI), hvor de fleste tilfælde varer under en time uden permanent hjerneskade \cite{Ritter2015,Sundhed.dk2014}. Flere tusinde danskere oplever TCI årligt, men det er sjældent, at den ramte selv opdager det, da symptomerne heraf er milde. \cite{Hjernesagen2015a,Academic2015} 
%
%De fysiologiske konsekvenser kommer til udtryk enten sensoriske og motoriske og er de hyppigst forekommende følger hos apopleksiramte. Disse kan bla. medføre problemer med balancen, udførsel af orienterede handlinger eller genkendelsesvanskeligheder. \cite{Kruuse2015a,DSfA2009}  De sensoriske og motoriske funktioner har indflydelse på hinanden, da der ofte anvendes sanser og motorik til udførsel af forskellige funktioner \cite{Nichols1997}\\
%Ud disse skader kan der også opstå psykiske lidelser som f.eks. depression eller angst, hvilket bl.a. går udover patientens lyst til at komme tilbage til sin normale hverdag. \cite{Muus2008}\\
%
%Sundhedsstyrelsen har udarbejdet et forløbsprogram for rehabilitering af patienter med hjerneskade, som er opdelt i fire faser. Fase to og tre omhandler især rehabiliteringen af patienten. \cite{Sundhedsstyrelsen2011a} Rehabiliteringsmulighederne varierer imellem de individuelle patienter ift. personens begrænsninger. Fælles for alle er dog, at patienterne skal have en kognitiv kapacitet til at følge instruktionerne under behandlingssessioner og fastholde læring fra session til session. Derudover kræves neurologisk kapacitet til at genskabe frivillig kontrol, samt motorisk kapacitet, hvis patienten skal opnå genskabelse af evt. tabte fysiske funktioner. \cite{Middaugh1989} 
%
%Biofeedback  er en rehabiliteringsmetode, som blev introduceret i slutningen af 1960\cite{Glanz1995,Prentice2007}. Det er en terapeutisk metode, der hjælper individet med at genoptræne fysiologiske aktiviteter og kropsfunktioner, der er blevet glemt eller gået tabt som følge af f.eks. apopleksi \cite{Prentice2007}. Metoden kan anvendes både før, under og efter udførelsens af øvelser \cite{Prentice2007, Giggins2013}. \\
%Der findes flere forskellige apparater og sensorer til at opfange fysiologiske signaler. Signalerne opfanges af apparatet eller sensoren, hvorefter signalet behandles og fortolkes af et system. Systemet kan herefter give feedback til patienten på baggrund af signalernes information. \cite{Prentice2007} Denne feedback leveres til patienten visuelt, auditivt og sensorisk.\\
%Der kan benyttes forskellige typer af sensorer til at opfange fysiologiske signaler fra patienterne, som kan deles ind i en fysiologisk og en biomekanisk del. \cite{Giggins2013} Fysiologisk biofeedback omfatter måling på forskellige kropslige systemer og kan f.eks. anvendes til patienter med balanceproblemer ved brug af elektromyografisk (EMG) feedback, hvor myoelektriske signaler omsættes til et signal til patienten, hvormed der kan opnås bevidsthed om svage muskler. Ved biomekanisk biofeedback måles der på generelle motoriske egenskaber såsom kroppens bevægelser og selve kropsholdningen.\cite{Giggins2013} Fordelen ved at benytte et accelerometer til at detektere apopleksipatienternes kropshældning er, at det kan måle patientens acceleration i en bestemt retning ift. tyngdekraften. For at få de bedste resultater ift. genkendelse af apopleksipatientens kropshældning placeres accelerometeret øverst på sternum \cite{Gjoreski2011}. Accelerometeret har derved formålet at advare apopleksipatienter, der kommer i ubalance, for at undgå faldulykker. \cite{Hjaelpemiddelbasen} \\
%
%Et biologisk signal skal behandles for at kunne give et feedback samt et digitalt output. For at kunne behandle et signal fra et accelerometer kræves der hhv. en forstærker, filtre, komparator samt ADC. Der kan anvendes andre komponenter til signalbehandling ift. hvad accelerometret skal benyttes til, men de nævnte vil blive benyttet i dette projekt.
%
%\section{Problemafgrænsning}
%Apopleksi er en sygdom, der har stor indflydelse på blodtilførslen til encephalon. Hvis tilstrømningen af blod er nedsat, kan der opstå både motoriske og sensoriske skader hos patienten, hvilket kan komme til udtryk som balanceproblemer. Balancen er vigtig for at kunne fungere i dagligdagen, da den sikrer at man holder kroppen oprejst og muliggør bevægelse uden fald. \cite{Nichols1997} Apopleksipatienter med balanceproblemer oplever en begrænsning i deres dagligdag, da de er afhængige af hjælp til daglige gøremål, som de før sygdommen selv kunne udføre. De oplever det som et brud på deres tidligere liv, hvilket påvirker deres identitet og livskvalitet.\cite{Sundhedsstyrelsen2010}
%
%For at begrænse de fysiske, og dermed også de personlige, følger mest muligt, er det essentielt at rehabiliteringen påbegyndes hurtigt efter apopleksitilfældet. Indenfor rehabilitering af balance tilbydes forskellige metoder, såsom platform feedback og passiv sensorisk stimulation. En anden mulighed ift. rehabilitering af balancen er biofeedback. Studier viser positive resultater med biomekanisk biofeedback, herunder inerti-sensorer, hvor der måles på kroppens generelle motoriske egenskaber. \cite{Giggins2013} For at biofeedback er en mulighed, er det en forudsætning, at patientens kognitive evner er tilstrækkelige til at kunne blive instrueret og kunne huske de indlærte øvelser fra gang til gang. \cite{Middaugh1989} Dette gør sig især gældende for den ældre befolkning, som systemet skal designes til, da det er denne befolkningsgruppe, der i højere grad rammes af apopleksi. \cite{Sundhedsstyrelsen2011}
%
%Det er interessant at undersøge, hvordan et system baseret på biomekanisk biofeedback kan designes således, at det vha. et accelerometer hjælper apopleksipatienter med at genoptræne deres balance. Det er essentielt at undersøge, om systemet kan designes sådan, at det i højere grad tillader patienterne at bidrage til deres egen rehabilitering ved at benytte visuel, sensorisk og/eller audio biofeedback. Det er muligt, at dette kan begrænse nogle af patienternes personlige følger, da kontakten med sundhedspersonale i forbindelse med rehabiliteringen kan begrænses, hvormed det normale hverdagsliv hurtigere kan genoptages. 
%
%\section{Problemformulering}
%Hvordan designes et biofeedback system således, at det hjælper apopleksipatienter under rehabilitering af balancen?