% !TeX spellcheck = da_DK
\subsection{Kravspecifikation}
I dette afsnit er der beskrevet, hvilke kravspecifikationer de enkelte komponenter i blokdiagrammet skal have for at kunne implementeres. De forskellige tolerance krav bestemmes på baggrund af et pilotforsøg som udføres snarest.

\subsubsection{Opsamling}
Blokken opsamling omfatter systemets sensor dvs. accelerometer samt en forstærker. Accelerometeret skal være lille, så det kan påsættes en patient uden at give fysiske begrænsninger. Desuden skal det være flerakset, så det bliver muligt at påsætte det i en optimal position, der giver mindst støj samt kunne måle accelerationen i både negativ og positiv retning. Derudover skal det kunne detektere hældningsgraden af patienten. Dette gøres ud fra tyngdekraften, hvor accelerometeret giver et output i mV alt efter påvirkningen fra tyngdekraften, kaldes også statisk acceleration. Dette output vil kunne omregnes til grader. Accelerometeret skal kunne detektere en hældningsgrad på $\pm$ 90$^{\circ}$, så det er muligt at detektere at patienten er faldet. 
\newpage

\subsubsection{Accelerometer}
\textbf{Krav:}
\begin{itemize}
\item Skal mindst have to output akser
\item Skal være under 10x10 cm i dimensionerne
\item Skal kunne måle statisk acceleration
\item Skal kunne måle acceleration i positiv og negativ retning
\item Skal kunne detektere en hældning fra 0 grader til $\pm$ 90$^{\circ}$ i forhold til tyngdekraften
\item Skal give et outputsignal  i form af elektrisk spænding målt i volt
\end{itemize}
\textbf{Tolerance:}
\begin{itemize}
\item Tolerance krav vil blive beskrevet på baggrund af pilotforsøg.
%\item Hældningen skal have en lille tolerance for ikke at risikere at patienten ikke vil falde og sikre at man ikke bevæger sig ud i den næste risikozone uden at der sker noget.
\end{itemize}

\subsubsection{Forstærker}
For at kunne adskille det ønskede signal fra diverse støjkomponenter skal signalet forstærkes således at der kan skelnes mellem disse når signalet filtres. Da signalet allerede ligger inden for arbejdsområdet, er det ikke nødvendigt at forstærke amplituden af signalet. En forstærkning af signalet med et gain på 1, vil gøre at noget af den støj der indgår i signalet vil blive frasorteret, da de to kredsløb vil blive adskilt. \\
\textbf{Krav:}
\begin{itemize}
\item Skal forstærke signalet med et gain på 1
\item Skal have en indgangsimpedans på over 1m$\Omega$
\end{itemize}
\textbf{Tolerance:}\\
Tolerance krav vil blive beskrevet på baggrund af pilotforsøg.
\begin{itemize}
\item Der må være en afvigelse i indgangsimpedans på 5 \%
\end{itemize}

\subsubsection{Filter}
Der skal anvendes et filter til at frafiltrere uønsket signaler. De ønskede biologiske signaler ligger i området 0 - 20 Hz. Det ønskes derfor at frafiltrere alle signaler der ligger udenfor dette spektrum. Dette gøres ved at benytte et lavpasfilter, der har en knækfrekvens ved 20 Hz. 

Der findes forskellige støjkilder der kan påvirke biologiske signaler, bl.a. støj fra elnettet og støj fra ledninger. Støj fra elnettet findes ved 50 Hz, derfor er det et krav at filteret har en stopbåndsfrekvens inden 50 Hz. Støj fra ledninger er lavfrekvent, og ligger i det ønskede spektrum. Derfor kan det være svært at identificere og frafiltrere. \\
\textbf{Krav:}
\begin{itemize}
\item Skal have en knækfrekvens på 20 Hz, så alle signaler over bliver dæmpet
\item Skal have en stopbåndsfrekvens på 45 Hz, så støj fra elnettet ikke påvirker signalet
\end{itemize}
\textbf{Tolerance:}
\begin{itemize}
\item På knækfrekvensen er der en tolerance på $\pm$ 2 Hz
\item På stopbåndsfrekvensen er der en tolerance på $\pm$ 2 Hz
\end{itemize}
\newpage

\subsubsection{Tilpasning}
For at tilpasse signalet til ADC'en, som har et arbejdsområde fra 0 - 3V, skal der benyttes en variabel forstærker. Denne skal sørge for at signalet ikke kommer udenfor arbejdsområdet.
\textbf{Krav:}
\begin{itemize}
\item Skal kunne dæmpe amplituden af signalet, hvis det ligger over 3V
\item Skal kunne forstærke amplitude af signalet, hvis det ligger under 0V
\end{itemize}
\textbf{Tolerance:}
\begin{itemize}
\item Kravene skal opfyldes
\end{itemize}

\subsubsection{Komparator}
Komparatoren skal kunne sammenligne inputsignalet med en referencespænding. Komparatoren skal kunne bestemme i hvilken grad og i hvilken retning patienten hælder. Der skal være flere tærskelværdier, så feedbacken kan gives i flere niveauer. Derudover skal der benyttes en switch, så det bliver muligt at ændre tærskelværdierne. Derved gøres systemet mere fleksibelt, og kan benyttes af patienter på balanceniveauer. Komparatoren skal kunne tænde for en diode og en vibrator, som afhænger hældningsgraden hos patienten. \\
\textbf{Krav:}
\begin{itemize}
\item Skal være en switch, så tærskelværdierne kan ændres
\item Forysningsspænding på 9 V, da vi skal bruge et 9 V batteri som forsyningsspænding
\item Skal ved en bestemt spænding tænde for en diode og vibrator
\end{itemize}
\textbf{Tolerance:}
\begin{itemize}
\item Der tolereres en afvigelse i tærskelværdien på 2 \%
\item Batteriet skal afgive en spænding på over 3 V
\end{itemize}

\subsubsection{Feedback}
Der skal være både visuel og sensorisk feedback i form af LED-dioder og vibratorer. Der skal være fem LED-dioder, en grøn, der indikerer midtpunkt, en gul diode på hver side af den grønne diode, en rød diode på hver side af den gule diode. Den gule diode skal begynde at lyse ved en bestemt hældningsgrad, der indikerer at patienten skal være opmærksom på sin balance. Den røde diode skal begynde at lyse ved en bestemt hældningsgrad, der indikerer at patienten er i fare for at falde. Dioderne til højre for den grønne diode, skal indikere at patienten falder til højre. Det samme  for venstre. Den sensoriske feedback skal vibrere ved en bestemt hældningsgrad, samtidig med de enkelte dioder lyser. Hvis den gule diode lyser, skal vibrationen være let og hvis den røde diode lyser skal vibrationen være moderat. Der skal placeres en vibrator på højre side og venstre side af patienten, der skal indikere i hvilken retning patienten falder. Spændingsforsyningen til feedbacksystem skal udgøres af 9 V batterier. \\
\textbf{Krav:}
\begin{itemize}
\item Skal have en spændingsforsyning på 9 V
\item Skal have en vibrator der har mindst to vibrationshastigheder
\item LED-dioderne og vibratorerne skal informere patienten om hvilken retning hældningen sker
\item LED-dioderne og vibratorerne skal informere patienten om risikoen for fald
\item LED-dioderne skal kunne lyse kraftigt nok til at at kunne ses på 5 meters afstand
\end{itemize}
\textbf{Tolerance:}
\begin{itemize}
\item Der accepteres af påbegyndelsen af feedbacken på $\pm$ 2 \%
\end{itemize}

\subsubsection{ADC}
Der anvendes en ADC i systemet, for at konvertere det analoge signal til digitalt. Den skal kunne sample det forstærkede signal. ADC'ens inputssignal skal ligge fra 0 til 3V, da systemet skal kunne benyttes fjerde semester, hvor der benyttes en mikrokontroller med dette arbejdsområde. Det anbefales, at en ADC der skal opsamle et signal fra en variabel forstærkning har en opløsning på 12-bit \cite{Zouridakis2003}. Båndbredden på accelerometeret bestemmer hvilken samplingsfrekvens, der skal benyttes. I følge Nyquists sætning skal der samples med minimun det dobbelte af det biologiske signal. I praksis samples der med minimum det tidobbelte. \\
\textbf{Krav:}
\begin{itemize}
\item Skal kunne modtage et inputssignal i intervallet 0 til 3 V.
\item Skal have en samplingsfrekvens på minimum det tidobbelte af accelerometerets båndbredde
\end{itemize}
\textbf{Tolerance:}
\begin{itemize}
\item Der accepteres ingen afvigelse ift. ADCen.
\end{itemize}

\subsubsection{USB-isolator}
Systemet skal være sikkert for patienten at benytte, hvilket USB-isolatoren har til formål at sikre. USB-isolatoren sikre at der ikke løber lækstrømme fra computeren ind i systemets kredsløb.\\
\textbf{Krav:}
\begin{itemize}
\item Skal have den samme outputspænding som inputspænding. 
\end{itemize}
\textbf{Tolerance:}
\begin{itemize}
\item Der accepteres ingen afvigelse ift. USB-isolatoren. 
\end{itemize}

\subsubsection{Computer}
Computeren er systemets digitale output. Det er brugerfladen for det fagkyndige personale og skal derfor kunne fremvise information omkring patienternes balance i form af grafer eller lignende. Det fagkyndige personale skal vha. af programmet kunne følge med i patienternes udvikling ift. balancen og programmet hvori graferne vises skal derfor have følgende krav. \\
\textbf{Krav:}
\begin{itemize}
\item Skal kunne fremvise en graf med information om patientens hældning i de enkelte øvelser. Herunder i hvor høj grad patienten har bevæget sig ud i risikozonerne. 
\item Skal kunne gemme data, så fagkyndigt personale kan følge med i patienternes udvikling ift. rehabiliteringen af balancefunktionen.
\item Skal være brugervenligt for det fagkyndige personale, dvs. designet af programmet skal være enkelt. 
\end{itemize}
\textbf{Tolerance:}
\begin{itemize}
\item Der accepteres ingen afvigelse ift. computeren og det program det fagkyndige personale skal anvende. 
\end{itemize}


