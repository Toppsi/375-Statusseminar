% !TeX spellcheck = da_DK
\subsection{Kravspecifikation}
I dette afsnit er der beskrevet, hvilke kravspecifikationer de enkelte komponenter i blokdiagrammet skal have for at kunne implementeres. De forskellige tolerance krav er bestemt på baggrund af pilotforsøg som ses under \fxnote{reference til pilotforsøg}

\subsubsection{Opsamling}
Blokken opsamling omfatter systemets sensor dvs. accelerometer samt forstærker. Det er nødvendigt at kunne måle udover de angivne grader således at feedbacken ikke stoppes, hvis patienten bevæger sig ud over den maksimale grænse på 13 grader.

\textbf{Accelerometer}
\textbf{Krav:}
\begin{itemize}
\item Skal kunne detektere en hældning fra 0 grader til $\pm$ 90 grader i forhold til tyngdekraften
\item Skal kunne detektere en spænding målt i volt ud fra accelerationen som finder sted i X-aksen
\end{itemize}

\textbf{Tolerance:}
Tolerance krav vil blive beskrevet på baggrund af pilotforsøg.
\begin{itemize}
\item Hældningen skal have en lille tolerance for ikke at risikere at patienten ikke vil falde og sikre at man ikke bevæger sig ud i den næste risikozone uden at der sker noget.
\end{itemize}

\textbf{Forstærker}
For at kunne adskille det ønskede signal fra diverse støjkomponenter skal signalet forstærkes således at der kan skelnes mellem disse når signalet filtres. Da signalet allerede ligger inden for arbejdsområdet, er det ikke nødvendigt at forstærke. Det kan dog alligevel forstærkes en smule, for at skabe en form for buffer der sikre at signalet kan filtreres optimalt. 
\textbf{Krav:}
\begin{itemize}
\item Forstærkeren skal forstærke signalet med et gain på 1
\item 
\end{itemize}

\textbf{Tolerance:}
Tolerance krav vil blive beskrevet på baggrund af pilotforsøg.
\begin{itemize}
\item Der skal være en lille tolerance, da filtreringen kun er en buffer. 
\end{itemize}

\subsubsection{Filter}
Der skal anvendes et filter til at filtrere uønsket støj. Signalet ligger fra 0-20Hz derfor ønskes det at alt signal over dette frafiltreres. 

\textbf{Krav:}
\begin{itemize}
\item Der skal anvendes et filter som filtrerer signaler over 20 Hz
\end{itemize}

\textbf{Tolerance:}
Tolerance krav vil blive beskrevet på baggrund af pilotforsøg.
\begin{itemize}
\item noget
\end{itemize}

\subsubsection{Forstærker}
\textbf{Krav:}
\begin{itemize}
\item noget
\end{itemize}

\textbf{Tolerance:}
Tolerance krav vil blive beskrevet på baggrund af pilotforsøg.
\begin{itemize}
\item noget
\end{itemize}

\subsubsection{ADC}
Der anvendes en ADC i systemet, for at konvertere det analoge signal til digitalt. Den skal kunne sample det forstærkede signal. ADCens inputssignal vil ligge fra 0 til 3V. Det anbefales, at en ADC der skal opsamle et signal fra en variabel forstærkning har en opløsning på 12-bit \cite{Zouridakis2003}.
\textbf{Krav:}
\begin{itemize}
\item Skal kunne modtage et inputssignal i intervallet 0 til 3 V.
\item Skal have en samplingsfrekvens på minimum 100 Hz. 
\end{itemize}

\textbf{Tolerance:}
\begin{itemize}
\item Der accepteres ingen afvigelse ift. ADCen.
\end{itemize}

\subsubsection{USB-isolator}
Systemet skal være sikkert for patienten at benytte, hvilket USB-isolatoren har til formål at sikre.
\textbf{Krav:}
\begin{itemize}
\item Skal have den samme outputspænding som inputspænding. 
\end{itemize}

\textbf{Tolerance:}
\begin{itemize}
\item Der accepteres ingen afvigelse ift. USB-isolatoren. 
\end{itemize}

\subsubsection{Computer}
Computeren er systemets digitale out. Det er brugerfladen for det fagkyndige personale og skal derfor kunne fremvise information omkring patienternes balance i form af grafer. Det fagkyndige personale skal vha. af programmet kunne følge med i patienternes udvikling ift. balancen og programmet hvori graferne vises skal derfor have følgende krav.
\textbf{Krav:}
\begin{itemize}
\item Skal kunne fremvise en graf med information om patientens hældning i de enkelte øvelser. Herunder i hvor høj grad patienten har bevæget sig ud i risikozonerne. 
\item Skal kunne gemme data, så fagkyndigt personale kan følge med i patienternes udvikling ift. balancen.
\item Skal være brugervenligt for det fagkyndige personale, dvs. designet af programmet skal være enkelt. 
\end{itemize}

\textbf{Tolerance:}
\begin{itemize}
\item Der accepteres ingen afvigelse ift. computeren og det program det fagkyndige personale skal anvende. 
\end{itemize}


