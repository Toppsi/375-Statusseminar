% !TeX spellcheck = da_DK
\section{Systemets funktionelle krav}
Systemets input er patienternes kropshældning, altså hvor meget vedkommende svajer i normal kropsstilling (anatomisk udgangsposition) og under udførelse af en bestemt øvelse. Systemet skal kunne konvertere informationer vedrørende patienternes kropshældning til visuel og sensorisk feedback samt give et digitalt output i form af grafer på en computer. Selve systemet skal anvendes til selvtræning i hjemmet, hvor to sværhedsgrader kan anvendes. De to sværhedsgrader indstilles ift. om patienten står i den normale kropsstilling og udfører en udvalgte træningsøvelse. Den pågældende træningsøvelse er valgt for i højere grad at udfordre patienternes balance, da kropsvægten fordeles anderledes ved øvelsen ift. den normale kropsstilling. Hvis patienternes kropshældning overskrider den normale grænse for krops svaj, der er seks til syv grader i lateral retning, vil den visuelle og sensoriske feedback gøre patienten opmærksom på dette samt retning af hældningen. Således kan systemet registrere, hvis patienten er i risiko for at falde og dermed fungere som en hjælp for apopleksipatienter. Udover den visuelle og sensoriske feedback har systemet et digitalt output, så patienternes øvelsesresultater kan behandles og gemmes på en computer. Den digitale del af systemet henvender sig derfor til fagkyndigt personale, der på denne måde kan følge med i udviklingen af patienternes balance.\\

De funktionelle krav:
\begin{itemize}
\item Systemet skal være brugervenligt, så det kan anvendes af apopleksipatienter og fagkyndigt personale.
\item Systemet skal kunne måle patienternes kropshældning og konvertere det til visuel og sensorisk feedback samt et digitalt output.
\item Systemet skal kunne give visuel og sensorisk feedback til patienten ved forskellig hældningsgrad ift. hvilken sværhedsgrad, patienten vælger.
\item Systemet skal kunne give et digitalt output, så det er muligt at behandle og gemme patienternes data på en computer.
\item Systemet skal indeholde en "switch" knap, så patienterne kan skifte mellem de to sværhedsgrader.
\end{itemize}


%Systemet har til formål at hjælpe apopleksipatienter med rehabiliteringen af deres balancefunktion i hjemmet. Dette gøres ved en træningøvelse, hvor systemet skal gå ind og advare patienten om risiko for fald under øvelsen. Måden patienten skal advares, er ved at måle kropshældningen under forsøget. Hvis patienten kommer til at hælde for meget, skal der være et biofeedback system, som advarer patienten om hvilken retning de er ved at falde til højre eller venstre. Feedbacksystemet udgøres af en visuel og en sensorisk del. %Der benyttes to feedback dele, da brugeren af systemet kan være dårligt seende, eller have kognitive problemer, som kan gøre det svært at opfange kun en form for feedback.   
%Derudover skal der være en digital feedback, som kan gemmes, så øvelsens resultater kan ses igen. 